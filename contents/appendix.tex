\chapter[附\hskip\ccwd{}\hskip\ccwd{}录]{{\heiti\zihao{3}附\hskip\ccwd{}\hskip\ccwd{}录}}

\section[\hspace{-2pt}作者在攻读硕士学位期间的论文和专利目录]{{\heiti\zihao{-3} \hspace{-8pt}作者在攻读硕士学位期间的论文和专利目录}}

%下面是盲审标记\cs{secretize}的用法,记得去\textsf{main.tex}开启盲审开关看效果:

% \circled{1}已发表论文

% \begin{enumerate}
%     \item \textbf{\secretize{XU X}}, \secretize{LIU K}, DAI P, et al. Joint task offloading and resource optimization in NOMA-based vehicular edge computing: A game-theoretic DRL approach[J]. Journal of Systems Architecture, 2023, 134: 102780. 影响因子: 5.836(2021), 4.497(5年) (中科院SCI 2区,对应本文第三章)
% 	\item \textbf{\secretize{许新操}}, \secretize{刘凯}, 刘春晖, 等. 基于势博弈的车载边缘计算信道分配方法[J]. 电子学报, 2021,49(5): 851-860. (EI 索引,CCF T1类中文高质量科技期刊,对应本文第三章)
% 	\item \textbf{ \secretize{XU X}}, \secretize{LIU K}, XIAO K, et al. Vehicular fog computing enabled real-time collision warning via trajectory calibration[J]. Mobile Networks and Applications, 2020, 25(6): 2482-2494. 影响因子: 3.077(2021), 2.92(5年) (中科院SCI 3区,对应本文第五章)
% \end{enumerate}
{
\small
\setlength{\baselineskip}{20pt}
\begin{enumerate}[label={[\arabic*]}, leftmargin=*]
\item Tang W, Zhou F, 导师 \secretize{Huang S}, 作者 \secretize{\textbf{Zhu X}}, et al. Feature re-embedding: Towards foundation model-level performance in computational pathology[C]//Proceedings of the IEEE/CVF Conference on Computer Vision and Pattern Recognition(CVPR). 2024: 11343-11352. 
\item 作者 \secretize{\textbf{Zhu X}}, 导师 \secretize{Huang S}, Tang W,  et al. Adaptive Learning of Instance Representatives in Dual Spaces for Medical Image Classification[J]. Neural Computing and Applications (中科院SCI二区,已完成第一次返修)
\item 作者 \secretize{\textbf{Zhu X}}, 导师 \secretize{Huang S}, Liu B, et al. Scan-invariant Mamba with Contrastive Learning in Computational Pathology[J]. IEEE Transactions on Medical Imaging(SCI一区,Submitted)
\item 导师 \secretize{黄晟}, 作者 \secretize{\textbf{朱翔}}, 刘倍言, 董佳俊, 吴治霖, 邱可真, 洪明坚, 葛永新, 徐玲, 杨梦宁. 一种基于提示分割任务的混合监督文本检测方法: CN118781606A[P]. 2024-10-15
\item 导师 \secretize{黄晟}, 方珩, 唐文浩, 作者 \secretize{\textbf{朱翔}}, 石华展, 杨俣豪, 官子焱, 洪明坚, 葛永新, 徐玲. 一种基于空间上下文感知的全视野数字切片图像分类方法: CN118570534A[P]. 2024-08-30
\item 导师 \secretize{黄晟},  张小先, 唐文浩, 作者 \secretize{\textbf{朱翔}}, 徐玲, 葛永新, 杨梦宁, 张小洪. 一种基于掩码的数字病理图像分类方法: CN116486159A[P]. 2023-07-25
\end{enumerate}
}


\section[\hspace{-2pt}作者在攻读硕士学位期间参与的科研项目]{{\heiti\zihao{-3} \hspace{-8pt}作者在攻读硕士学位期间参与的科研项目}}

{
\small
\setlength{\baselineskip}{20pt}
\begin{enumerate}[label={[\arabic*]}, leftmargin=*]
\item 国家自然科学基金面上项目,少样本学习特征生成与鲁棒性关键技术研究
% (No. 62176030)
\item 重庆市自然科学基金面上项目,文本描述协同的双向生成式少样本学习研究
\end{enumerate}
}

\newpage
\section[\hspace{-2pt}学位论文数据集]{{\heiti\zihao{-3} \hspace{-8pt}学位论文数据集}}

\begin{table}[h]
% \resizebox{\textwidth}{!}{%
\begin{tabular}{|cccccccccccc|}
\hline
\multicolumn{4}{|c|}{\heiti{关键词}}             & \multicolumn{4}{c|}{\heiti{密级}}   & \multicolumn{4}{c|}{\heiti{中图分类号}}                                    \\ \hline
\multicolumn{4}{|c|}{\begin{tabular}{c} 数字病理图像分类;Mamba架构;\\多实例学习;弱监督学习; \end{tabular}} & \multicolumn{4}{c|}{公开} & \multicolumn{4}{c|}{TP} \\ \hline
\multicolumn{3}{|c|}{\heiti{学位授予单位名称}} & \multicolumn{3}{c|}{\heiti{学位授予单位代码}}    & \multicolumn{3}{c|}{\heiti{学位类别}}  & \multicolumn{3}{c|}{\heiti{学位级别}}        \\ \hline
\multicolumn{3}{|c|}{\secretize{重庆大学}}     & \multicolumn{3}{c|}{\secretize{10611}}       & \multicolumn{3}{c|}{专业硕士}  & \multicolumn{3}{c|}{硕士}          \\ \hline
\multicolumn{4}{|c|}{\heiti{论文题名}}            & \multicolumn{4}{c|}{\heiti{并列题名}} & \multicolumn{4}{c|}{\heiti{论文语种}}                                     \\ \hline
\multicolumn{4}{|c|}{\begin{tabular}{c}基于Mamba的高分辨率医学\\病理图像分类研究\end{tabular}}               & \multicolumn{4}{c|}{/}   & \multicolumn{4}{c|}{汉语} \\ \hline
\multicolumn{3}{|c|}{\heiti{作者姓名}}     & \multicolumn{3}{c|}{\secretize{朱翔}}         & \multicolumn{3}{c|}{\heiti{学号}}    & \multicolumn{3}{c|}{\secretize{202224131038t}} \\ \hline
\multicolumn{6}{|c|}{\heiti{培养单位名称}}                                      & \multicolumn{6}{c|}{\heiti{培养单位代码}}                                   \\ \hline
\multicolumn{6}{|c|}{\secretize{重庆大学}}                                        & \multicolumn{6}{c|}{\secretize{10611}}                                    \\ \hline
\multicolumn{3}{|c|}{\heiti{学科专业}}     & \multicolumn{3}{c|}{\heiti{研究方向}}        & \multicolumn{3}{c|}{\heiti{学制}}    & \multicolumn{3}{c|}{\heiti{学位授予年}}       \\ \hline
\multicolumn{3}{|c|}{软件工程} & \multicolumn{3}{c|}{计算机视觉}         & \multicolumn{3}{c|}{3年}     & \multicolumn{3}{c|}{\secretize{2025年}}        \\ \hline
\multicolumn{3}{|c|}{\heiti{论文提交日期}}   & \multicolumn{3}{c|}{\secretize{2025年6月}}     & \multicolumn{3}{c|}{\heiti{论文总页数}} & \multicolumn{3}{c|}{\pageref{LastPage}}         \\ \hline
\multicolumn{3}{|c|}{\heiti{导师姓名}}     & \multicolumn{3}{c|}{\secretize{黄晟}}          & \multicolumn{3}{c|}{\heiti{职称}}    & \multicolumn{3}{c|}{教授}          \\ \hline
\multicolumn{6}{|c|}{\heiti{答辩委员会主席}}                                     & \multicolumn{6}{c|}{}                                      \\ \hline
\multicolumn{12}{|c|}{\heiti{\begin{tabular}{c} 电子版论文提交格式\\ 文本(\checkmark) 图像() 视频()音频()多媒体()其他()\end{tabular}}}                              \\ \hline
\end{tabular}%
% }
\end{table}

