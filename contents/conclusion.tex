\chapter[\hspace{0pt}总结与未来展望]{{\heiti\zihao{3}\hspace{0pt}总结与未来展望}}\label{section 7}
\removelofgap
\removelotgap

本章内容共分为两节,\hyperref[section5: 总结]{第一节}对本文研究内容与方法进行总结;\hyperref[section5: 未来展望]{第二节}会阐述本文提出方法存在的局限之处,同时对后续的研究方向展开展望。


\section[\hspace{-2pt}总结]{{\heiti\zihao{-3} \hspace{-8pt}总结}}\label{section5: 总结}


本研究聚焦于 Mamba 架构在病理图像分类领域的应用,旨在解决该架构在面对病理图像分类任务时所暴露出的一系列问题。
病理图像分类任务由于实例数目众多,常导致计算资源消耗巨大,同时实例间关系建模也往往不够充分。
Mamba 架构虽在一定程度上有效应对了这些挑战,
但其仍存在无法充分捕获二维空间关系,以及与病理图像多实例学习的序列不变性存在冲突的问题。​

为解决上述问题,实现 Mamba 架构在病理图像领域的高效适配,
本研究从多扫描模式融合 Mamba 和扫描不变性 Mamba 两个独特角度展开深入研究:

(1)在基于多扫描 Mamba 的高分辨率病理图像分类研究方面,
鉴于Mamba一维序列建模所造成图像空间结构捕获能力不足的问题,研究提出了基于空间信息增强实现的多扫描 Mamba 的多实例学习框架。
该框架创新性地设计了多个扫描模式分支作为Mamba的输入,提出了区域螺旋扫描和网格扫描等新型扫描模式,显著提升了 Mamba 对局部连续性、旋转不变性以及同一序列不同方向空间关联信息的捕获能力。
同时,通过设计基于分块局部注意力的2D空间上下文感知模块,进一步强化序列对2D空间上下文的感知能力,
通过区域内信息整合,极大地提高了Mamba对空间能力的捕捉,
有力地推动了Mamba模型与病理图像分类任务的适配进程。
在3个WSI分析子任务和7个数据集上进行的大量实验充分证明,综合多种扫描模式分支并引入空间信息,
能够切实解决Mamba从一维实例序列中空间建模不充分的问题,显著提高模型性能,同时起到一定缓解数字病理图像分类任务中使用Mamba模型时出现的建模倾向差异的作用。​

(2)在基于扫描不变性Mamba的高分辨率病理图像分类研究方面,
%利用多扫描模式信息融合缓解Mamba建模能力与病理图像任务间差异的思路固然直接,
%但其仅是一种妥协的方案,高度依赖扫描模式和集成形式。
%基于此,
针对Mamba作为序列相关模型,与病理图像多实例学习的序列不变性存在冲突的问题,研究提出了基于对比学习策略的扫描不变 Mamba 的多实例学习框架。
为引导 Mamba 学习到不同实例扫描序列中一致的包特征,精心设计了双分支实例序列对比学习框架,强制在两个极端不同的扫描序列之间,通过相同的 Mamba 获得的包表示及分类 logits 值保持一致性。
同时,还设计了一种基于实例评估的序列对比增强方法,不仅增加了对比学习的难度,还迫使 Mamba 在输入序列差异极大的情况下,学习对 WSI 图像形成一致的表示和决策,成功赋予了 Mamba 良好的序列不变建模和判别实例识别能力。
同样在3个WSI分析子任务和12个基准上开展的广泛系统实验结果显示,SMC-MIL 能够在让 Mamba 继承其优异计算效率和建模能力的同时,有效降低模型对输入序列顺序的敏感性,专注于实例无序关系建模,挖掘出更具显著性判别的特征,从而更好地适配病理图像分类任务。​

总体而言,本研究提出的M$^2$S-MIL模型和SMC-MIL模型,分别从多扫描模式Mamba和扫描不变性Mamba两个维度,对数字病理图像分类任务中Mamba的应用进行了全面且深入的剖析与研究。
M$^2$S-MIL 模型通过综合多种扫描模式的特征信息,并引入空间上下文感知模块,显著提升了 Mamba 对 2D 空间信息的捕获能力,有效缓解了原本Mamba一维序列建模所造成图像空间结构捕获能力不足的问题;
SMC-MIL 模型则进一步突破了扫描顺序对 Mamba 建模能力的束缚,成功弥合了原本 Mamba 的序列敏感性与病理图像多实例学习中序列不变性假设之间的冲突,使 Mamba 架构更加契合病理图像任务,同时提高了模型挖掘更具判别性区域的能力,进而提升了模型性能。​


\section[\hspace{-2pt}未来展望]{{\heiti\zihao{-3} \hspace{-8pt}未来展望}}\label{section5: 未来展望}

本文分析了在 Mamba 架构适配病理图像分类任务方面的挑战,以数据的视觉特征和任务特点为切入点,从多扫描信息融合和构造扫描不变性Mamba两个角度入手,提出了基于空间信息增强的多扫描Mamba模型和基于差异化序列对比的扫描不变Mamba模型来解决高分辨率病理图像分类问题,取得了一定成果,但仍存在诸多可拓展与深入研究的方向。​

在扫描优化层面,未来可进一步探索更高效的扫描模式设计。现有的区域螺旋扫描和网格扫描虽已取得良好效果,但随着病理图像复杂性的不断增加,可能需要研发更具针对性、能够更精准捕捉图像特征的扫描模式。例如,可以结合病理图像中不同组织类型、病变特征的分布特点,设计自适应的扫描模式,使 Mamba 能够更快速、准确地获取关键信息。​

对于对比学习策略,也有很大的提升空间。当前的双分支实例序列对比学习框架和基于实例评估的序列对比增强方法已发挥重要作用,但可以尝试引入更复杂的对比学习算法,如结合生成对抗网络(GAN)的思想,让 Mamba 在与生成器的对抗过程中,学习到更具鲁棒性和判别性的特征表示。同时,在对比学习的过程中,如何更好地平衡模型的训练效率和性能提升,也是需要深入研究的问题。​

从应用拓展角度来看,本文的研究依旧聚焦于多实例学习范式中将离线的特征进行特征聚合。未来可以迁移至特征提取,或者尝试打破范式,探索在线特征提取与实例聚合的病理图像分类方法。

此外,与临床实践的结合也是未来研究的重要方向。目前的研究更多停留在实验室环境下的模型性能验证,而将这些模型真正应用到临床诊断中,还需要解决一系列实际问题,如模型的可解释性、与现有临床诊断流程的兼容性等。通过与临床医生合作,深入了解临床需求,能够使模型的改进更贴合实际应用,为病理诊断提供更有力的支持。​

在数据层面,随着医疗数据的不断积累,如何更好地利用大规模病理图像数据训练 Mamba 模型也是未来研究的重点。一方面,可以探索更有效的数据增强方法,在不改变图像真实病理特征的前提下,增加数据的多样性,提高模型的泛化能力;另一方面,如何从海量数据中筛选出对模型训练最有价值的数据,避免冗余数据对模型性能的影响,也是需要解决的问题。​

最后,随着人工智能技术的不断发展,跨领域技术融合将为病理图像分类带来新的机遇。例如,将 Mamba 架构与知识图谱技术相结合,利用知识图谱中丰富的医学知识,引导 Mamba 更好地理解病理图像中的特征和关系,从而提高分类的准确性和可靠性。总之,未来在 Mamba 架构应用于病理图像分类领域的研究中,仍有广阔的探索空间,有望通过不断的技术创新和应用拓展,为病理诊断领域带来更显著的变革。