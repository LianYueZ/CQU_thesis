\chapter[致\hskip\ccwd{}\hskip\ccwd{}谢]{{\heiti\zihao{3}致\hskip\ccwd{}\hskip\ccwd{}谢}}

% 这里用盲审环境包裹致谢,在开启盲审开关时,环境内部的内容不予渲染。
\begin{secretizeEnv}

行文至此,我的毕业论文即将画上句号,回顾这段求学历程,心中满是感慨。
转眼间,从本科到硕士,已在重庆度过了七年,我的求学之路也告一段落。临近道别之际,借此机会给每一个支持和帮助过我的人们说一声感谢。

首先,感谢我的导师黄晟教授。在整个研究生阶段,导师给予了我无微不至的指导和关怀。是他严谨的工作态度与悉心的教导,让我不断进步,最终完成了这篇毕业论文。
在我遇到困难和挑战时,导师总是耐心倾听、指导并鼓励我,帮助我面对困难。在此,我向导师黄晟教授表示最诚挚的感谢!

其次,我想感谢师兄师姐师弟师妹和同门们。三年来,我们一起吃饭、一起学习、一起进步。一个人是做不好科研,与人交流合作,才能明白自己的缺陷与局限。
特别地,我想感谢张译师兄,如果不是张译师兄,我不会有认识到黄老师及课题组其他家人的机会,是他的温柔让我感受到兄长的关怀与照顾,进而产生想加入该课题组的想法。我还想特别感谢周锋涛师兄、唐文浩师兄和石华展师弟。
与两位师兄的交流探讨常常让我受益匪浅,他们学术上的见解让我获益良多;与师弟的交流也让清晰自身的缺陷和问题。与三位的思想碰撞是我感觉三年科研生活中思想的高光。
此外,还得特别感谢我的室友兼同门兼好兄弟严杰轩。他在本文的撰写过程中提供了很多建议与帮助,没有他的指导,我的文字表达将更加一塌糊涂。

另外,感谢我的家人。无论是在生活中还是在学业上,他们始终是我坚强的后盾,一直默默支持着我,尤其是我的父亲,他用他的身躯帮我抵挡了太多社会的压力与淤泥,使我能够在象牙塔般的校园中安心学习。
感谢我的女朋友,与我分享快乐,遇到挫折时给予我鼓励,帮助我渡过难关,不断前行。

感谢足球,让我学会了团结与坚持,让我遇到了很多生动的人,收获了珍贵的感情。在重大虎溪操场奔跑的这几年,我把开心、低落、满足、遗憾、高兴、幽怨等情绪全都留在了这里。我会时刻记住那些瞬间,带着它们奔跑在未来更广阔的绿茵场上。

感谢自己,虽然没有追上寄予厚望的自己但也按时长大。心态更加成熟,眼界更加开阔。感谢自己做出的每一个选择、每一份努力,造就了现在的自己。

最后,感谢百忙之中参与评阅和答辩的各位专家、教授。

% \vfill
\vspace*{2em}
\begin{flushright}
{\CJKfontspec{STXingkai} \Large 朱翔} \hspace*{3.5em}
\\  \hspace*{\fill} \\
{二〇二五年五月\hspace*{1em}于重庆}
\end{flushright}
\end{secretizeEnv}